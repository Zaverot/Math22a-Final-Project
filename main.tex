\documentclass{article}
\usepackage{amsmath}
\usepackage{graphicx}
\usepackage{caption}

\title{Math22 Final Project}
\date{Fall 2021}
\author{Andy Li\\ \texttt{andy\_li@college.harvard.edu} \and Ethan Tran\\ \texttt{etran@college.harvard.edu}}

\begin{document}

\maketitle

\pagebreak

\tableofcontents

\section{Overview}
Networks are a common graph structure in mathematics with vertices connected by edges, usually with the purpose of visualizing relationships between objects.
The use of network graphs have been extended into other fields, and this project will focus on the use of networks in sociology.
Here, networks have nodes that represent agents, which are connected by links (edges) that signify some social interaction or tie between the two agents.
Further detail comes in the form of node degrees (number of connections) and strength (weight of connections) to represent networks (Porter, 1082). 
 
A large component of networks lies in communities, which are densely connected subgroups of nodes that in the context of sociology represent extensive human interaction.
These communities are discovered through community detection algorithms, which have important real world uses such connecting people with common interests over social media
Even with the wonderous applications of community networks in sociology, they are still built on the fundamentals of graphs which utilize linear algebra for construction. 
Thus begs the guiding question:
How does linear algebra assist in community detection and community network graphing?

\section{Network Science}

\subsection{What is Network Science?}
The academic study of networks is often referred to as \textbf{Network Science}. This field was born from what is known as the "Small World Problem,"
discussed in such landmark papers as Stanley Milgram's 1967 thought experiment of the same name. This problem is best represented by the following question:
"Given any two people in the world, person X and person Z, how many intermediate acquaintance links are required before X and Z are connected?" (Milgram, 1967). This experiment has been
pondered extensively, and even carried out by multiple universities, as can be seen in Columbia University's "An Experimental Study of Search in Global Social Networks" (Ribiero 2020). In this experiment,
searchers tasked around 60,000 individuals with contacting one of 18 individuals from 13 different countries by forwarding a single email to acquaintances. Analyzing the path these emails took,
these Columbia researchers were able to map out an extensive network of interconnected individuals. This is Network Science\textemdash the science of identifying the structure of networks and extrapolating information
from them.

\subsection{What is a Network?}
A \textbf{Network} is any set of objects or concepts connected by some relationship or interaction. This profoundly vague definition is best expounded on through examples. In the social sciences, a network
might be a group of individuals that are related by family, friendship, work, location, academic co-authorship or any other social bond. In biology, the interconnected parts of the brain or even the set of proteins responsible for metabolism are networks. 
The hierarchy of objects and classes in an object-oriented programming project is a network. Even words can be conceived of as a network when connected by meaning, part of speech, or pronunciation. As can be seen, a network is an abstract tool to envision 
interconnected groups of objects, individuals, and even ideas.   

\subsection{What Can Networks Teach Us?}
The analysis of networks can yield valuable for a multitude of fields. Studying networks can help identify the organizational principles behind them\textemdash patterns, properties, etc.. These organizational principles can form the basis for statistical models
that allow for a better understanding of the behaviors and relationships present in the network. These models essentially map out expected relationships and behaviors for a given group, enabling easy identification of outlier occurences. Finally, most pertinent to this paper,
networks can be represented with mathematical \textbf{Graph Theory} and analyzed with a suite of algorithms utilized for \textbf{Community Detection}.

\section{Graph Theory}

\subsection{What is Graph Theory?}
Graph theory is the study of \textbf{Graphs}, which are the  mathematical representations of the real systems referred to as "networks." Generally, graphs are composed of nodes, also known as vertexes, and edges.
Nodes represent individual members of a network\textemdash people, roads, products, websites, objects in a piece of software, and so on and so forth. Edges represent the connections between them.
Nodes and Edges can also have any combination of various attributes (Ribeiro 2020). Some examples for edges are:

\bigskip
\begin{description}
  \item[\textbf{Weight}] - Duration of a phone call, distance of a road\dots
  \item[\textbf{Ranking}] - Best Friend, Second Best Friend\dots
  \item[\textbf{Type}] - Friend, Relative, Co-worker (Often represented with different edge colors)  
\end{description}

\bigskip

\noindent Nodes can also have attributes such as:
\bigskip
\begin{description}
  \item[\textbf{Type}] - Nationality, Age, Sex (Often represented with different node colors)
\end{description}
\bigskip
In the most common definition of a graph $G(V, E)$, there is a set $V$ of nodes/vertices/points, and set of edges $E$ such that \\$E \subseteq \{\{x, y\} | x, y \in V \text{ and } x \neq y\}$ (Bender and Williamson 7). 
Various modifications can be applied to this definition to create different types of graphs. 
For example, directed graphs, in which edges have orientation to and from certain nodes, have different definitions of edges $E$ compared to undirected graphs, which follow the definition above. 
The direction of edges in a directed graph can represent a variety of phenomenon. 
For example, in a network graph of a road system, the direction of edges can indicate one-way and two-way access between two roads. If one was mapping phone calls between individuals, 
the direction of edges can establish who is calling and who is receiving.  

\subsection{Adjacency Matrices}
We will define adjacency matrices, which are common algebraic representations of graphs in Graph Theory (Biggs 7).
Recall from earlier that a graph $G$ has a set of vertices or nodes $V$, and a set of edges $E$ to connect them.
These matrices show which nodes $v \in V$ are adjacent, or connected, to each other.

\bigskip

\textbf{Definition 1:} The adjacency matrix of a network is the $n \times n$ matrix $A$ whose entries $a_{ij}$ are given by
\[   
a_{ij} = 
     \begin{cases}
       1 & \text{if node } v_i \text{ and node } v_j \text{ are adjacent, or connected by an edge} \\
       0 & \text{otherwise} \\
     \end{cases}
\]

\bigskip

An example adjacency matrix is shown here:

\bigskip

\noindent\begin{minipage}{.5\textwidth}
\centering
\includegraphics[scale=0.4]{"s3graph.pdf"}
\captionof{figure}{A graph with three nodes}
\label{fig:fig1}            
\end{minipage}%
\begin{minipage}{.5\textwidth}
\centering
\vspace{1.2cm}
$\begin{bmatrix}
0 & 1 & 1\\
1 & 0 & 1\\
1 & 1 & 0
\end{bmatrix}$
\vspace{1.1cm}

\captionof{figure}{Associated adjacency matrix}
\label{fig:fig2}            
\end{minipage}

\bigskip

Notice how all diagonal entries are $0$ as each node is not connected with itself, 
and all non-diagonal entries are $1$ as each node is connected to every other node.
A more complex example is shown below

\bigskip

\noindent\begin{minipage}{.5\textwidth}
\centering
\includegraphics[scale=0.4]{s4graph.drawio.pdf}
\captionof{figure}{A graph with four nodes}
\label{fig:fig3}            
\end{minipage}%
\begin{minipage}{.5\textwidth}
\centering
\vspace{1cm}
 $\begin{bmatrix}
0 & 1 & 1 & 0\\
1 & 0 & 0 & 1\\
1 & 0 & 0 & 1\\
0 & 1 & 1 & 0
\end{bmatrix}$
\vspace{0.88cm}

\captionof{figure}{Associated adjacency matrix}
\label{fig:fig4}            
\end{minipage}

\bigskip

Adjacency matrices can be easily used to construct graphs (both directed and undirected) and vice versa.
We can note a few characteristics of adjacency matrices. Firstly, for undirected graphs, the adjacency matrix will be symmetric. 
This is because if two nodes are connected, then both $a_{ij}$ entries in the matrix will be $1$ as direction doesn't matter, and $0$ otherwise if they are not connected.
On the other hand, for directed graphs, adjacency matrices are not necessarily symmetric. 
As the $a_{ij}$ entry will only be $1$ if there exists an edge from $i$ to $j$, if there is not a edge going in reverse then the "symmetric" value in the matrix will have a value of $0$.
Usually, the diagonal values of an adjacency matrix will all have the value $0$, as most graphs will not have nodes that are connected with themselves.

Adjacency matrices become more complicated in the environment of community networks. 
Instead of using $1$ to represent the presence of a edge between two nodes in the matrix, the nonzero value can be greater or less than $1$ to represent the weight of edge.
In community networks, this weight represents how close two agents relate to or interact with each other, and is critical in community detection algorithms, which will be elaborated on below (Porter 1086).    

\bigskip

An example adjacency matrix along with its associated graph is shown here:

\bigskip 

\bigskip

\section{Community Detection}

\subsection{What is Community Detection?}

Community Detection is the process by which researchers can identify groups within a network.
Community Detection is performed by researchers to map out the relationships between isolated nodes, members, or objects within the community. 
By gathering and ordering network members by the strength of their relationships to each other one can discover communities\textemdash otherwise known as groups, clusters, cohesive subgroups, or modules. 
This information can yield valuable insights. For example, community detection can reveal the key figures and leaders within a group, around which other members gather.
The results of community detection are often represented through graphs of nodes and edges, and are generated through the application of algorithms to networks as represented by adjacency matrices.

\subsection{CD Algorithm}
There are many different algorithms used for community detection on graphs, each with their own approaches and required data structures. 
For the purposes of this project, we will focus on an algorithm that utilizes the eigenvectors of matrices. 
Specifically, this algorithm tries to find densely connected subgraphs in a graph by calculating the leading non-negative eigenvector of the modularity matrix of the graph (Chandal, Kabre, Aggarwal 66). 

\bigskip 

\noindent This algorithm focuses on modularity, which is a measure the strength of division of a network into modules (also called groups, clusters or communities). The modularity matrix of the graph is an important aspect of this algorithm to detect communities. 


\section{Sample Community Network}
In this simple example, we have interviewed five people about who they eat with most frequently. 
They will rank their peers from 1-4, with 1 representing the person they would be least likely to eat with and 4 representing the person they would most be likely to eat with. 
On a five by five matrix, with zeroes or ones on the diagonal depending on if a person eats with themselves or not, we will convert these values to an adjacency matrix.
This matrix will display the strength of connection according to how often they eat with someone.
We will proceed to generate a directed graph with the use of this data and the CD Algorithm.

\pagebreak
\section*{References}

\bigskip

\noindent “Communities in Networks” by Mason A. Porter, Jukka-Pekka Onnela, and Peter J. Mucha

\bigskip 

\noindent "An Introduction to Community Detection" by Lei Tang and Huan Liu

\bigskip 

\noindent "Community Detection Algorithms" by Thamindu Dilshan Jayawickrama

\bigskip 

\noindent "Algebraic Graph Theory" by Norman Biggs

\bigskip 

\noindent "Lists, Decisions and Graphs. With an Introduction to Probability" by Edward A. Bender and S. Gill Williamson

\bigskip 

\noindent "Community Structure Detection" by Amar Chandole, Ameya Kabre, and \linebreak Atishay Aggarwal

\bigskip

\noindent "An Introduction to Network Science" by Pedro Ribeiro

\bigskip

\noindent Milgram, S. "The small world problem." Psychol. Today 2, 60–67 (1967)

\bigskip

\noindent "Two Decades of Network Science: as seen through the co-authorship network of network scientists" by Roland Molontay and Marcell Nagy



%http://web.cs.ucla.edu/~yzsun/classes/2017Winter_CS249/Slides/Clustering1.pdf

%http://ai-nlp.info.uniroma2.it/basili/didattica/BigData/005_Intro_to_CommunityDetection.pdf
%finding community detection https://www.sciencedirect.com/science/article/pii/S1568494616300242
%https://www.nature.com/articles/s41598-018-23932-z
%https://towardsdatascience.com/community-detection-algorithms-9bd8951e7dae
%https://superoles.files.wordpress.com/2015/09/n-biggs-algebraic-graph-theory-1993.pdf
%https://stackoverflow.com/questions/9471906/what-are-the-differences-between-community-detection-algorithms-in-igraph
%https://www.dcc.fc.up.pt/~pribeiro/aulas/ns2021/1_introduction.pdf
%http://files.diario-de-bordo-redes-conecti.webnode.com/200000013-211982212c/AN%20EXPERIMENTAL%20STUDY%20by%20Travers%20and%20Milgram.pdf
%https://arxiv.org/pdf/1908.08478.pdf

\end{document}