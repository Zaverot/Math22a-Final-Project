\documentclass{article}
\usepackage{amsmath}
\usepackage{amsthm}
\usepackage{graphicx}
\usepackage{caption}
\usepackage{enumitem}
\setlist[description]{leftmargin=\parindent,labelindent=\parindent}
\newtheorem{theorem}{Theorem}[section]
\newtheorem{corollary}{Corollary}[theorem]
\newtheorem{lemma}[theorem]{Lemma}
\DeclareMathOperator{\Tr}{Tr}

\title{Community Detection and Linear Algebra}
\date{Fall 2021}
\author{Andy Li\\ \texttt{andy\_li@college.harvard.edu} \and Ethan Tran\\ \texttt{etran@college.harvard.edu}}

\begin{document}

\maketitle

\pagebreak

\begin{center}
 {\huge Cover Letter}
\end{center}
 
 \bigskip 
 
We, authors of this project, Andy Li and Ethan Tran, would like to graciously thank Kevin Lin, Aris Zhu, Zoe Cooper, and Professor Dusty Grundmeier for their articulate and constructive advice on how to improve our draft. 
Clarity of ideas, additional examples, and more complete definitions were the main areas to refine in our draft. 
To accomplish this, we vastly improved each of our definitions with more formal mathematical notation and diagrams wherever possible, especially for definitions such as 1, 3, 6 which are major components of our project. 
Several of our student feedback comments expressed confusion with the idea of networks and communities. 
We worked to resolve this confusion by adding additional explanations for both using information from the Ribiero presentation. 
Historical background for the study of social network's using Milgram's 1967 "Small World" thought experiment was added as well as this concept is fundamental to understanding the field and helps establish our motivating question within an existing academic endeavor. 
We had many comments regarding a need for visualizations of our definitions, prompting us to add graphs, matrices, and other examples below our definitions. 
We have also had comments regarding the need for refined citations, which were rectified by a thorough rewriting of in-text and bibliography citations to make them adhere to a consistent Chicago, author-date style. We also include page numbers in our in-text citations to increase clarity of source. 
Our guiding problem was edited to more closely resemble a practical application of community detection techniques (ie. the analysis of a hypothetically plausible real-life group made up of Harvard Students). 
We believe that this will serve as a more concrete basis off which we can launch our paper, rather than our previous, more abstract introduction, which primarily focused on the definitions of a community. 
We have also fixed several instances of inconsistent mathematic notation, particular in our discussion of the partitioning function used to calculate $s$. 
Then we retooled our definition of the Null Model Matrix $P$ to use more specific mathmatic terminology and eliminate confusion surrounding how it is derived. 
We also added a new definition 9 for trace, which is used in our immediately preceding definition and discussion.
Once again, we thank all of our readers for their time and insight, and hope that our paper is an interesting introduction to community detection! 


\bigskip


 
 \bigskip 
 
\pagebreak 

\tableofcontents

\section{Motivating Problem}
Imagine that we have gathered an assortment of Harvard students, and we want to identify two things: 
\bigskip
\begin{enumerate}
  \item Who among this group of students are friends with each other?
  \item Are there any smaller groups of students that are more friendly with each other than with those outside their smaller group? 
        In other words, are there any friend groups or cliques?
\end{enumerate}
\bigskip

How might we answer these questions? One approach would be to analyze this group of students as a network.

\section{Overview}
Networks are a common graph structure in mathematics with vertices connected by edges, usually with the purpose of visualizing relationships between objects.
The use of network graphs have been extended into other fields, and this project will focus on the use of networks in sociology.
Here, networks have nodes that represent agents, which are connected by links (edges) that signify some social interaction or tie between the two agents.
Further detail comes in the form of node degrees (number of connections) and strength (weight of connections) to represent networks (Porter 2009, 1082). 
 
A large component of networks lies in communities, which are densely connected subgroups of nodes that in the context of sociology represent extensive human interaction.
These communities are discovered through community detection algorithms, which have important real world uses such connecting people with common interests over social media
Even with the wonderous applications of community networks in sociology, they are still built on the fundamentals of graphs which utilize linear algebra for construction. 
Thus begs the guiding question:
How does linear algebra assist in community detection and community network graphing?

\section{Network Science}

\subsection{What is Network Science?}
The academic study of networks is often referred to as \textbf{Network Science}. This field was born from what is known as the "Small World Problem,"
discussed in such landmark papers as Stanley Milgram's 1967 thought experiment of the same name. This problem is best represented by the following question:
"Given any two people in the world, person X and person Z, how many intermediate acquaintance links are required before X and Z are connected?" (Milgram 1967, 2). This experiment has been
pondered extensively, and even carried out by multiple universities, as can be seen in Columbia University's "An Experimental Study of Search in Global Social Networks" (Ribiero 2020, 9). In this experiment,
searchers tasked around 60,000 individuals with contacting one of 18 individuals from 13 different countries by forwarding a single email to acquaintances. Analyzing the path these emails took,
these Columbia researchers were able to map out an extensive network of interconnected individuals. This is Network Science\textemdash the science of identifying the structure of networks and extrapolating information
from them.

\subsection{What is a Network?}
A \textbf{Network} is any set of objects or concepts connected by some relationship or interaction. 
This profoundly vague definition is best expounded on through examples. In the social sciences, a network might be a group of individuals that are related by family, friendship, work, location, academic co-authorship or any other social bond. 
In biology, the interconnected parts of the brain or even the set of proteins responsible for metabolism are networks. 
The hierarchy of objects and classes in an object-oriented programming project is a network. 
Even words can be conceived of as a network when connected by meaning, part of speech, or pronunciation. As can be seen, a network is an abstract tool to envision 
interconnected groups of objects, individuals, and even ideas.   

\subsection{What Can Networks Teach Us?}
The analysis of networks can yield valuable for a multitude of fields. Studying networks can help identify the organizational principles behind them\textemdash patterns, properties, etc.. These organizational principles can form the basis for statistical models
that allow for a better understanding of the behaviors and relationships present in the network. These models essentially map out expected relationships and behaviors for a given group, enabling easy identification of outlier occurences. Finally, most pertinent to this paper,
networks can be represented with mathematical \textbf{Graph Theory} and analyzed with a suite of algorithms utilized for \textbf{Community Detection}.

\section{Graph Theory}

\subsection{What is Graph Theory?}
Graph theory is the study of \textbf{Graphs}, which are the mathematical representations of the real systems referred to as "networks." 
In this paper, the terms graphs and networks are interchangeable as we are using mathematical graphs to model networks. 
Thus, definitions that apply to graphs will also apply to networks. 
A more formal definition is given below: 

\bigskip

\noindent \textbf{Definition 1:} A \textbf{graph} $G$, usually denoted as $G(V, E)$, has two parameters, $V$ and $E$. 

\indent \begin{description}
  \item[\textbf{$V$}] - The set of all vertices, also known as \textbf{nodes}. 
  \item[\textbf{$E$}] - The set of all \textbf{edges}, which connect vertices in the graph. 
\end{description}

An example graph is shown here:

\bigskip

\noindent\begin{minipage}{.5\textwidth}
\centering
\includegraphics[scale=0.4]{"egraph.drawio.pdf"}
\captionof{figure}{Graph $G$ with four nodes}
\label{fig:fig21}            
\end{minipage}%
\begin{minipage}{.5\textwidth}
\centering

\begin{description}
  \item[\textbf{$V$}] - Set of nodes given by $\{1, 2, 3, 4\}$
  \item[\textbf{$E$}] - Set of all edges given \\by $\{(1, 2), (1, 3), (2, 3), (3, 4)\}$
\end{description}


\label{fig:fig22}            
\end{minipage}

\bigskip

\noindent Nodes represent individual members of a network\textemdash people, roads, products, websites, objects in a piece of software\textemdash which will also be referred to in this paper as agents of a network. 
Nodes and Edges can also have any combination of various attributes (Ribeiro 2020, 64-65). Some examples for edges are:

\bigskip
\indent \begin{description}
  \item[\textbf{Weight}] - Duration of a phone call, distance of a road\dots
  \item[\textbf{Ranking}] - Best Friend, Second Best Friend\dots
  \item[\textbf{Type}] - Friend, Relative, Co-worker (Often represented with different edge colors)  
\end{description}

\bigskip

\noindent Nodes can also have attributes such as:
\bigskip
\begin{description}
  \item[\textbf{Type}] - Nationality, Age, Sex (Often represented with different node colors)
\end{description}
\bigskip

Additionally, there is a set of terms used to describe the number of edges originating from a given node (Ribeiro 2020, 67).

\bigskip

\textbf{Definition 2:} The \textbf{degree} of a given node is the total number of outgoing or incoming edges incident on it. 
In directed graphs, the degree can be further separated into \textbf{outdegree}, the number of edges originating from the node (start of the edge directed towards the target node), and \textbf{indegree},
the number of edges that terminate at the node (end of the edge directed from the initial node).

\bigskip

\subsection{Adjacency Matrices}
We will define adjacency matrices, which are common algebraic representations of graphs in Graph Theory (Biggs 2010, 7).
Recall from earlier that a graph $G$ has a set of vertices or nodes $V$, and a set of edges $E$ to connect them.
These matrices show which nodes $v \in V$ are adjacent, or connected, to each other. 
Each node $v$ will have some distinct assignment such that node $v_i$ corresponds to a row and column $i$ in the adjacency matrix. 
The ordering of these notations is insignificant, as the assignments only serve to distinguish nodes from each other and the adjancency matrix tracks which nodes are adjacent. 

\bigskip

\textbf{Definition 3:} The \textbf{adjacency matrix} of a network is the $n \times n$ matrix $A$ whose entries $a_{i,j}$ are given by
\[   
a_{i,j} = 
     \begin{cases}
       1 & \text{if node } v_i \text{ and node } v_j \text{ are adjacent, or connected by an edge} \\
       0 & \text{otherwise} \\
     \end{cases}
\]

\bigskip

An example graph, along with its adjacency matrix, is shown here:

\bigskip

\noindent\begin{minipage}{.5\textwidth}
\centering
\includegraphics[scale=0.4]{"s3graph.pdf"}
\captionof{figure}{Graph $G_1$ with three nodes}
\label{fig:fig1}            
\end{minipage}%
\begin{minipage}{.5\textwidth}
\centering
\vspace{1.2cm}
$\begin{bmatrix}
0 & 1 & 1\\
1 & 0 & 1\\
1 & 1 & 0
\end{bmatrix}$
\vspace{1.1cm}

\captionof{figure}{Adjacency matrix $A_1$}
\label{fig:fig2}            
\end{minipage}

\bigskip

The graph $G_1$ has three nodes, with $V = \{ v_1, v_2, v_3\}$.
Each column and row in the adjacency matrix corresponds to a node, so column $1$ and row $1$ correspond to $v_1$, and so on.
As $v_1$ is adjacent (connected) to $v_2$ and $v_3$, entries $a_{1,2}$ and $a_{1,3}$ in adjacency matrix $A_1$ have entry $1$. 
A similar process applies for nodes $v_2$ and $v_3$. 
Notice how all diagonal entries are $0$ as each node is not connected with itself.

\bigskip 
\noindent A more complex example is shown below:

\bigskip

\noindent\begin{minipage}{.5\textwidth}
\centering
\includegraphics[scale=0.4]{s4graph.drawio.pdf}
\captionof{figure}{Graph $G_2$ with four nodes}
\label{fig:fig7}            
\end{minipage}%
\begin{minipage}{.5\textwidth}
\centering
\vspace{1cm}
 $\begin{bmatrix}
0 & 1 & 1 & 0\\
1 & 0 & 0 & 1\\
1 & 0 & 0 & 1\\
0 & 1 & 1 & 0
\end{bmatrix}$
\vspace{0.88cm}

\captionof{figure}{Adjacency matrix $A_2$}
\label{fig:fig8}            
\end{minipage}

\bigskip

\noindent Notice in this example that there is no connection between nodes $v_1, v_4$, and $v_2, v_3$ and thus the associated entries in the adjacency matrix are $0$.

\bigskip 
A few generalizations can be made. The first is that by construction of the adjacency matrix, they will always be square matrices.
Furthermore, the dimensions of the $n \times n$ matrix are that $|V| = n$, as there is a column and row for each node $v \in V$.
In the context of community network graphs, which we are representing in this paper, connections between an node and itself are often omitted for sake of clarity and/or insignificance to the study.
Thus, the diagonal of our adjacency matrices will always be zero, matricies of this type are referred to as being "hollow". 
Additionally, these matrices will always be symmetric, as connections (relationships) between nodes are symmetric, meaning for a given adjacency matrix $A$, the entries $a_{i,j} = a_{j,i}$.

\bigskip 

The complexity of adjacency matrices increases with the presence of weighted graphs and directed graphs which change the aforementioned properties and definitions. 
Although these modifications are significant to real-world studies, they are beyond the scope of this project, which will focus on unweighted, undirected graphs.


\bigskip 


\subsection{Degree Matrices}

\bigskip

While it is important to consider which nodes in a graph are connected through adjacency matrices, another important metric, especially for community detection, is how many connections a particular node has.
This is where degree matrices come in, which represent how many edges are connected to a particular node $v_i \in V$, or essentially how many adjacent nodes $v_i$ has (Chung, Lu, and Vu, 2003, 100).

\bigskip 

\textbf{Definition 4:} Given a graph $G(V, E)$, its degree matrix is the $n \times n$ matrix $D$ with entries $d_{i,j}$ given by
\[   
d_{i,j} = 
     \begin{cases}
       \text{deg} (v_i) & \text{if } i = j \\
       0 & \text{otherwise} \\
     \end{cases}
\]
In this case, deg$(v_i)$ is the total number of edges that are connected to the node $v_i$, per Definition (2) from earlier.

\bigskip 

An example graph, along with its degree matrix, is shown here:

\bigskip

\noindent\begin{minipage}{.5\textwidth}
\centering
\includegraphics[scale=0.4]{s4graph.drawio.pdf}
\captionof{figure}{Graph $G_2$ with four nodes}
\label{fig:fig3}            
\end{minipage}%
\begin{minipage}{.5\textwidth}
\centering
\vspace{1cm}
 $\begin{bmatrix}
2 & 0 & 0 & 0\\
0 & 2 & 0 & 0\\
0 & 0 & 2 & 0\\
0 & 0 & 0 & 2
\end{bmatrix}$
\vspace{0.88cm}

\captionof{figure}{Degree matrix $D_2$}
\label{fig:fig4}            
\end{minipage}

\bigskip

Notice how in the graph, each node $v\in V$ has exactly two connections and thus every entry $d_{i, i}$ along the diagonal has entry $2$.


\bigskip

\noindent A more complex example is shown below.

\bigskip

\noindent\begin{minipage}{.5\textwidth}
\centering
\includegraphics[scale=0.4]{s6graph.drawio.pdf}
\captionof{figure}{Graph $G_3$ with six nodes}
\label{fig:fig5}            
\end{minipage}%
\begin{minipage}{.5\textwidth}
\centering
\vspace{0.5cm}
$\begin{bmatrix}
2 & 0 & 0 & 0 & 0 & 0\\
0 & 3 & 0 & 0 & 0 & 0\\
0 & 0 & 2 & 0 & 0 & 0\\
0 & 0 & 0 & 3 & 0 & 0\\
0 & 0 & 0 & 0 & 3 & 0\\
0 & 0 & 0 & 0 & 0 & 1
\end{bmatrix}$
\vspace{0.5cm}

\captionof{figure}{Degree matrix $D_3$}
\label{fig:fig6}            
\end{minipage}

\bigskip

A few generalizations can be made. The first is that by construction of the degree matrix, they will always be diagonal matrices in which the only non-zero entries are along the main diagonal. 
Similar to adjacency matrices, the dimensions of the $n \times n$ matrix are that $|V| = n$, as there is a column and row for each node $v \in V$.
One might notice that degree matrices are the "opposite" of adjacency matrices: while adjacency matrices have all entries $0$ along the diagonal, degree matrices have all entries $0$ everywhere else. 
The relationship between adjacency and degree matrices can be defined below:

\begin{lemma}
Every entry $d_{i,i}$ along the diagonal of the degree matrix of a network is given by $\sum_j a_{i,j}$ where $a_{i,j}$ is an entry in the adjacency matrix of the network.
\end{lemma}

\begin{proof}
Assume that we have some network with corresponding adjacency matrix $A$. 
By definition of adjacency matrix, each row $i$ contains a series of entries ($0$ or $1$) indicating whether or not the node $v_i$ is connected to the node $v_j$ represented in column $j$. 
This means that the degree $k_i$ of a node $v_i$ to be $ k_{i} = \sum_j A_{i,j} $.
As each entry of $1$ represents an edge attached to the node $v_i$, summing all the entries in the row through $\sum_j$ gives the degree of $v_i$. 
Thus, $d_{i,i} = \sum_j a_{i,j}$.

\end{proof}

\noindent Degree matrices will an important part of detecting communities in networks.

\bigskip 

\section{Community Detection}

\subsection{What is Community Detection?}

Community Detection is the process by which researchers can identify groups within a network in which individuals' membership in particular cliques are yet to be ascertained. 
Community Detection is performed by researchers to map out the relationships between isolated nodes, members, or objects within the community (Aggarwal et al. 2017, 5). 
By gathering and ordering network members by the strength of their relationships to each other one can discover communities\textemdash otherwise known as groups, clusters, cohesive subgroups, or modules (Aggarwal et al. 2017, 4). 
This information can yield valuable insights. For example, community detection can reveal the key figures and leaders within a group, around which other members gather.
The results of community detection are often represented through graphs of nodes and edges, and are generated through the application of algorithms to networks as represented by adjacency matrices (Aggarwal et al. 2017, 6-12).

\subsection{Laplacian Matrices}

A \textbf{Laplacian Matrix} is a more complex graph representation which is used to find many useful properties of the graph (Weisstein 2021). More importantly, Laplacian graphs are critical to our community detection algorithm.

\bigskip 

\textbf{Definition 5:} Given a graph $G(V, E)$, its Laplacian matrix $L$ is given by $L = D - A$ where $D$ is the graph's degree matrix and $A$ is the graph's adjacency matrix. From our prior definition of adjacency matrices, $A$ has entries of $0$ for its main diagonal, and only has entries of $1$ or $0$ otherwise. The entries of $L$ are defined as follows:
\[   
L_{i,j} = 
     \begin{cases}
       \text{deg} (v_i) & \text{if } i = j \\
       -1 & \text{if } i \neq j \text{ and} \text{ node } v_i \text{ is adjacent to} \text{ node } v_j\\
       0 & \text{otherwise} \\
     \end{cases}
\]

\bigskip 

\noindent A Laplacian Matrix example is shown below, with graphs and all associated matrices included.

\bigskip 

\noindent\begin{minipage}{.5\textwidth}
\centering
\includegraphics[scale=0.4]{s6graph.drawio.pdf}
\captionof{figure}{Graph $G_3$ with six nodes}
\label{fig:fig9}            
\end{minipage}%
\begin{minipage}{.5\textwidth}
\centering
\vspace{0.5cm}
$\begin{bmatrix}
2 & 0 & 0 & 0 & 0 & 0\\
0 & 3 & 0 & 0 & 0 & 0\\
0 & 0 & 2 & 0 & 0 & 0\\
0 & 0 & 0 & 3 & 0 & 0\\
0 & 0 & 0 & 0 & 3 & 0\\
0 & 0 & 0 & 0 & 0 & 1
\end{bmatrix}$
\vspace{0.5cm}

\captionof{figure}{Degree matrix $D_3$}
\label{fig:fig10}            
\end{minipage}

\bigskip 

\noindent\begin{minipage}{.5\textwidth}
\centering
\vspace{0.5cm}
$\begin{bmatrix}
0 & 1 & 0 & 0 & 1 & 0\\
1 & 0 & 0 & 1 & 1 & 0\\
0 & 0 & 0 & 1 & 1 & 0\\
0 & 1 & 1 & 0 & 0 & 1\\
1 & 1 & 1 & 0 & 0 & 0\\
0 & 0 & 0 & 1 & 0 & 0
\end{bmatrix}$
\vspace{0.5cm}
\captionof{figure}{Adjacency Matrix $A_3$}
\label{fig:fig13}            
\end{minipage}%
\begin{minipage}{.5\textwidth}
\centering
\vspace{0.5cm}
$\begin{bmatrix}
2 & -1 & 0 & 0 & -1 & 0\\
-1 & 3 & 0 & -1 & -1 & 0\\
0 & 0 & 2 & -1 & -1 & 0\\
0 & -1 & -1 & 3 & 0 & -1\\
-1 & -1 & -1 & 0 & 3 & 0\\
0 & 0 & 0 & -1 & 0 & 1
\end{bmatrix}$
\vspace{0.5cm}

\captionof{figure}{Laplacian matrix $L_3$}
\label{fig:fig114}            
\end{minipage}

\bigskip 

Laplacian matrices are an important component of traditional graph partitioning methods, in which a graph was divided along lines into particular sections.
Each of these sections was a subset of the graph, with nodes that were connected more often to each other than to nodes outside of the section, forming a community. 
Graph division was performed using the eignvalues of the Laplacian matrices. While this method is effective, it is not without flaws. 
Often, graph partitioning only divided graphs into a small number of sections (often two), which would be ineffective for needing to find many communities within a graph.
Additionally, the partitioning algorithm did not always find the optimal divisions in the graph to make, which produced inaccurate communities.
Thus, for the purposes of this paper, we will focus on a more effective community detection method involving modularity matrices.

\bigskip 

\section{Community Detection Algorithm based \\ on Modularity Matrices}
This algorithm tries to find densely connected subgraphs in a graph by calculating the leading non-negative eigenvector of the modularity matrix of the graph (Aggarwal et al. 2017, 66). 


\subsection{The Modularity Matrix}
When finding communities within a given network, or graph, it is important to note that nodes within communities should have more links between themselves than the average number of edges between any two nodes in the network. 
This is because communities, by definition, have nodes that are more intra-connected (within the community) rather than inter-connected (between communities).

This concept is known as modularity, which measures the strength of division of a network into communities (Newman 2006, 103). 
Graphs with high modularity have stronger and more distinct communities. 
For simplicity in our case, we will consider modularity as dividing a network into two communities.

\bigskip

\textbf{Definition 6:} The \textbf{modularity} $Q$ of a set of communities is defined as: 

\bigskip

\indent $Q = m - P$ where $m$ is the total number of edges within the network and $P$ is an $n \times n$ "null model" matrix that gives the expected number of edges between any two vertices. 
  Q represents the fraction of edges that fall within each of community 1 or 2, minus the expected number of edges within each of community 1 or 2 (Newman 2006, 103).

\bigskip 

\noindent Thus the larger the modularity $Q$, the more that a graph is divided into two communities. 
An example of this concept is shown below:

\bigskip

\noindent\begin{minipage}{.5\textwidth}
\centering
\includegraphics[scale=0.4]{hqgraph.drawio.pdf}
\captionof{figure}{A network with high \\modularity $Q$}
\label{fig:fig23}            
\end{minipage}%
\begin{minipage}{.5\textwidth}
\centering

\includegraphics[scale=0.4]{lqgraph.drawio.pdf}
\captionof{figure}{A network with low \\modularity $Q$}


\label{fig:fig24}            
\end{minipage}

\bigskip

\textbf{Definition 7:} The \textbf{null model} matrix $P$ is a matrix created for the sake of comparison with another matrix, most often the adjacency matrix $A$. 
Essentially, $P$ is a theoretical or "ideal" adjacency matrix for a graph $G'$, which is derived from a graph $G$ that has as its adjacency matrix $A$. 
$G'$ has the same set of vertices $V$ as $G$, but its set of edges $E'$ are distributed randomly across its nodes $V$ using a uniform probability distribution rather than between set vertices in $G$.   

\bigskip 

$P$ follows the same basic constraints as $A$, meaning that $P$ can be assumed to be both symmetric (ie. $P = P^T$) and hollow (ie. $P_{i,i} = 0$). 
The sum of the entries of $P$ is equal to the sum of the entries in $A$, meaning:

$$\sum P_{i,j} = \sum A_{i,j}$$

\bigskip

Secondly, $P$ must have the same degree distribution as the real world network (adjacency matrix $A$), meaning:

\bigskip

$$\sum P_{ij} = k_{i}$$
(Note that if this is satisfied, so is the equation above)

\bigskip

where $k_{i}$, the degree of vertex $i$ is calculated using:

\bigskip

$$k_{i} = \sum_{j} A_{i,j}$$.

\bigskip

\noindent Now, assuming a random uniform distribution of the networks total edges among the vertices of $P$, 
we can now pursue the goal of our chosen community detection algorithm: to maximize the modularity of our graph partition, 
which produces the most accurate communities.

\bigskip

\subsection{Calculating Modularity}
Given $P_{i,j}$, we can define a formula for modularity $Q$ below (Newman 2010, 5):

$$ Q = \dfrac{1}{2m}\sum_{ij} [A_{i,j} - P_{i,j}]\delta (g_i, g_j) $$ 

$A_{i,j}$ gives the actual number of edges between vertices $v_i$ and $v_j$ per definition of the adjacency matrix.
As defined earlier, $P_{i,j}$ is the expected number of edges. 
Let $m$ be the total number of e  dges in the graph, given by $m = |E|$. 
We can define $g_i$ to be the community (tightly connected group of nodes that represents a real life cluster of people, ideas, etc) to which vertex $i$ belongs, which is important when used in conjunction with the Kronecker Delta function $\delta_{i,j}$.

\bigskip

\textbf{Definition 8:} The Kronecker Delta function $\delta_{i, j}$ (Newman 2010:7):  
\[   
\delta_{i,j} = 
     \begin{cases}
       1 & \text{if } i = j \\
       0 & \text{if } i \neq j \\
     \end{cases}
\]

Thus, the presence of $\delta (g_i, g_j)$ is our formula is critical.
If an edge exists in a community $g_i$, then $\delta (g_i, g_j)$ will be equal to $\delta (g_i, g_i)$ which will be $1$.
Otherwise, $\delta (g_i, g_j)$ will be $0$, and thus the entire term in the summation will be $0$ which will lower the overall modularity sum.
The quantity $\frac{1}{2m}$ is conventional and has no effect in the maximization of modularity $Q$.

\bigskip

\noindent Thus, the only component we need to further define and calculate is the null model matrix $P$.


\bigskip

\begin{lemma}
By following the definition of our null model matrix and placing edges at random for $P$, we have that 

$$ P_{i,j} = \dfrac{k_i k_j}{2m} $$

\end{lemma}

\begin{proof}

From our previously defined constraints on $P$, the edges between vertices will be placed entirely at random.
Thus, the probability that one end of an edge attaches to a vertex $v_i$ is dependent on the expected degree $k_i$ based on the definition of $P$
Additionally, both ends of the edge have independent probabilities in which vertices they attach to.

As edges are uniformly distributed, the expected number of edges $P_{i,j}$ between vertices $v_i$ and $v_j$ is given by $f(k_i) \times f(k_j)$, where $f$ is the Uniform probability density function.
Because edges are symmetric as defined from before, the same function $f$ is used for $k_i$ and $k_j$. 
Thus, from our requirement equation earlier, we have that:

$$ \sum_{j = 1}^{n} P_{i,j} = f(k_i) \sum_{j = 1}^{n} f(k_j) = k_i $$

for all $i$, by definition of the Uniform distribution. 
This means that $f(k_i) = C k_i$ for some constant $C$. Thus, we can apply our earlier equation to get

$$ 2m = \sum_{i, j} P_{i,j} = C^2 \sum_{i, j} k_i k_j = (2mC)^2$$

Thus, $C = \dfrac{1}{\sqrt{2m}}$ which gives us 

$$ P_{i,j} = \dfrac{k_i k_j}{2m} $$

\end{proof}

\bigskip

For simplicity, we will initially divide the network into two communities, with the ability to continue dividing using a recursive method.
Let us define an index vector $\vec{s}$ that contains $n$ elements, with each entry having a value of $1$ if the vertex corresponding to that entry is in the first community and $-1$ if it is in the second community.
Using this index vector, we can update our Kronecker delta function $\delta$ using the entries of $\vec{s}$. 
Let $s_i$ be the entry in $\vec{s}$ corresponding to the $v_i$th node. We have:

$$\delta (g_i, g_j) = \dfrac{(s_i s_j + 1)}{2}  $$

\bigskip 

Using this, we can update our original formula for $Q$:

$$ Q = \dfrac{1}{4m}\sum_{ij} [A_{i,j} - P_{i,j}](s_i s_j + 1) $$

or 

$$ Q = \dfrac{1}{4m}\sum_{ij} [A_{i,j} - P_{i,j}](s_i s_j) $$

using a simplification from equation $\sum P_{i,j} = \sum A_{i,j}$. 

\bigskip 

We can convert this into a matrix form as:

$$ Q = \dfrac{1}{4m}s^T B s $$

in which $\vec{s}$ is our index vector and $B$ is our modularity matrix, where $B = A - P$.

\bigskip

Unlike in standard spectral partitioning, which utilizes the Laplacian matrix, our algorithm uses the modularity matrix $B$.
As the eigenvalues of $B$ are not all necessarily all of of one sign, we must constrain the entries $s_1 \dots s_n$ of $\vec{s}$, 
which is a linear combination of the normalized eigenvectors $u$ of $B$. 
Each entry $s_i \in \vec{s}$ is calculated from entry $u_i \in u$ as follows:

\bigskip

\indent \[s_i = 
  \begin{cases}
    1 & \text{if } u_{i} \geq 0, \\
    -1 & \text{if } u_{i} < 0.   
  \end{cases} \]

\bigskip

From the definition of our modularity matrix $B$, the elements of the leading eigenvector (corresponding to largest eigenvalue)
measure how firmly each vertex belongs to one of the two partitions being created during this given application of the spectral partitioning process (Newman 2010, 7).
We select the leading eigenvector, which has the most positive magnitude, as we attempting to maximize the modularity of the entries within our partition groups, and as shown in our definition for $Q$, 
modularity is maximized when the entries of $B$ are maximized (Newman 2010, 7).

For most community networks, however, we want to divide the network into more than just $2$ subcommunities.
This is more complex, as our equation for modularity $Q$ from above can't be directly applied to the initial pair of partitioned communities.
Thus, we need to find the increase in modularity of the graph, $\Delta Q$, when a community is further divided using a variation of the formula.
This means that we find the difference between the modularities of the network before and after dividing a community into a subcommunity. 

\bigskip 

Let us define $G$ to be the set of vertices in our community which we plan on dividing, and $n_G$ the total number of vertices in the community.
To divide the community into $c$ subcommunities, we need a new index matrix $S'$ of dimensions $n_G \times c$.
This means that $S'_{i,j} = 1$ if vertex $g_i \in G$ is in subcommunity $j$ and $S'_{i,j} = 0$ otherwise. 
We can derive a new formula for the the change in modularity $\Delta Q$:

\begin{equation} \label{eq1}
\begin{split}
\Delta Q & = Q_\text{ after subdivision } - Q_\text{ before subdivision } \\
& = \sum_{i, j \in G} \sum_{k=1}^{c} B_{i,j} S_{i,k} S_{j, k} - \sum_{i, j \in G} B_{i,j} \\
 & = \sum_{k=1}^{c} \sum_{i, j \in G} [ B_{i, j} - \delta_{i, j} \sum_{l \in G} B_{i,l}]S_{i,k} S_{j, k} \\
 & = \Tr (S'^TB^{(G)}S')
\end{split}
\end{equation}

\bigskip

In particular, $\Tr (S'^TB^{(G)}S')$ is the trace of the matrix $(S'^TB^{(G)}S')$, which is defined as follows:

\bigskip 

\textbf{Definition 9:} The \textbf{trace} of a matrix $m$ is defined as the sum of the eigenvalues of the matrix $m$ (Weisstein 2021). In our formula, the trace notation is used a shorthand simplification as we are summing eigenvalues.

\bigskip

\noindent In this formula, $B^{(G)}$ is a $n_G \times n_G$ modularity matrix with entries given by

$$ B^{(G)}_{i, j} = B_{i,j} - \delta_{i, j} \sum_{l \in G} B_{i,l}$$

Notice that this formula for modularity change is very similar to the formula calculating modularity for the simple two-community partition from above.
If we substitute $S'$ for $S$ and $B^{(G)}$ for $B$ in our original modularity formula, we get the modularity change for the entire network and thus our $\Delta Q$ formula is correct.
By recursively applying this new formula, we can now separate networks into more than two communities.

\bigskip

Putting these proofs and definitions together, we can finally form our community detection algorithm (Newman 2010, 13-16).

\subsection{The Community Detection Algorithm}

\begin{enumerate}
  \item Compute the altered modularity matrix $B^{(G)}$ of the community $G$; we know that $B^{(G)} = B$ if $G$ is the entire network
  \item Compute the leading eigenvector of $B^{(G)}$. Use it to compute the index vector $\vec{s}$ and the modularity change $\Delta Q$.
  \item If $\Delta Q > 0$, then apply the procedure on the two communities created by the partition as indicated by $\vec{s}$
  \item If $\Delta Q = 0$, then take as the modularity matrix $B^\prime$, the submatrix of $B$ corresponding to the vertices of $G$, and compute the leading
        eigenvector, index vector $\vec{s^\prime}$, and modularity change $\Delta Q^{\prime}$. Note that by construction 0 is always an eigenvalue of the modularity matrix  
  \item If $\Delta Q^{\prime} > 0$, then apply the procedure on the two communities created by the partition as indicated by $s^\prime$, using $B^\prime$ as the modularity of the full network.
  \item If $\Delta Q^{\prime} = 0$, then terminate the algorithm.
\end{enumerate}

\pagebreak
\section{References}

\bigskip

\noindent Aggarwak, Atishay and Chandole, Amar and Kabre, Ameya. "Community Structure Detection." University of California at Los Angeles, 2017.

\bigskip

\noindent Biggs, Norman. "Algebraic Graph Theory: Second Edition." Cambridge University Press, 1993.

\bigskip 

\noindent Bender, Edward A. and Williamson, S. Gill. "Lists, Decisions and Graphs. With an Introduction to Probability." University of California at San Diego, 2010.

\bigskip 

\noindent Chung, Fan and Lu, Linyuan and Vu, Van. "Spectra of random graphs with given expected degrees." University of Pennsylvania, 2003.

\bigskip

\noindent Jayawickrama, Thamindu Dilshan. "Community Detection Algorithms." Towards Data Science, 2021.

\bigskip 

\noindent Milgram, S. "The small world problem." Psychology Today, no. 2 (1967): 60–67.

\bigskip

\noindent Molontay, Roland and Nagy, Marcell. "Two Decades of Network Science: as seen through the co-authorship network of network scientists." Cornell University, 2020.

\bigskip 

\noindent Newman, M. E. J. "Modularity and Community Structure in Networks." PNAS, 2006.

\bigskip

\noindent Porter, Mason A. and Onnela, Jukka-Pekka and Mucha, Peter J. "Communities in Networks." Cornell University, 2009.

\bigskip 

\noindent Ribeiro, Pedro. "An Introduction to Network Science." Universidade Do Porto, 2020. 

\bigskip

\noindent Tang, Lei and Liu, Huan. "Social Media Analysis and Recommending Systems: An Introduction to Community Detection." Morgan and Claypool, 2010.

\bigskip

\noindent Weisstein, Eric W. "Laplacian Matrix." From MathWorld, 2021. Last accessed December 13, 2021, https://mathworld.wolfram.com/LaplacianMatrix.html.

\bigskip

\noindent Weisstein, Eric W. "Matrix Trace." From MathWorld, 2021. Last accessed December 13, 2021, https://mathworld.wolfram.com/MatrixTrace.html.

%http://web.cs.ucla.edu/~yzsun/classes/2017Winter_CS249/Slides/Clustering1.pdf

%http://ai-nlp.info.uniroma2.it/basili/didattica/BigData/005_Intro_to_CommunityDetection.pdf
%finding community detection https://www.sciencedirect.com/science/article/pii/S1568494616300242
%https://www.nature.com/articles/s41598-018-23932-z
%https://towardsdatascience.com/community-detection-algorithms-9bd8951e7dae
%https://superoles.files.wordpress.com/2015/09/n-biggs-algebraic-graph-theory-1993.pdf
%https://stackoverflow.com/questions/9471906/what-are-the-differences-between-community-detection-algorithms-in-igraph
%https://www.dcc.fc.up.pt/~pribeiro/aulas/ns2021/1_introduction.pdf
%http://files.diario-de-bordo-redes-conecti.webnode.com/200000013-211982212c/AN%20EXPERIMENTAL%20STUDY%20by%20Travers%20and%20Milgram.pdf
%https://arxiv.org/pdf/1908.08478.pdf
%https://mathworld.wolfram.com/LaplacianMatrix.html
%https://www.ncbi.nlm.nih.gov/pmc/articles/PMC1482622/ Newman

\end{document}