\documentclass{article}
\usepackage{amsmath}
\usepackage{graphicx}
\usepackage{caption}
\usepackage{enumitem}
\setlist[description]{leftmargin=\parindent,labelindent=\parindent}

\title{Math22 Final Project}
\date{Fall 2021}
\author{Andy Li\\ \texttt{andy\_li@college.harvard.edu} \and Ethan Tran\\ \texttt{etran@college.harvard.edu}}

\begin{document}

\maketitle

\pagebreak

\tableofcontents

\section{Motivating Problem}
Imagine that we have gathered an assortment of Harvard students, and we want to identify two things: 
\bigskip
\begin{enumerate}
  \item Who among this group of students are friends with each other?
  \item Are there any smaller groups of students that are more friendly with each other than with those outside their smaller group? 
        In other words, are there any friend groups or cliques?
\end{enumerate}
\bigskip

How might we answer these questions? One approach would be to analyze this group of students as a network.

\section{Overview}
Networks are a common graph structure in mathematics with vertices connected by edges, usually with the purpose of visualizing relationships between objects.
The use of network graphs have been extended into other fields, and this project will focus on the use of networks in sociology.
Here, networks have nodes that represent agents, which are connected by links (edges) that signify some social interaction or tie between the two agents.
Further detail comes in the form of node degrees (number of connections) and strength (weight of connections) to represent networks (Porter, 1082). 
 
A large component of networks lies in communities, which are densely connected subgroups of nodes that in the context of sociology represent extensive human interaction.
These communities are discovered through community detection algorithms, which have important real world uses such connecting people with common interests over social media
Even with the wonderous applications of community networks in sociology, they are still built on the fundamentals of graphs which utilize linear algebra for construction. 
Thus begs the guiding question:
How does linear algebra assist in community detection and community network graphing?

\section{Network Science}

\subsection{What is Network Science?}
The academic study of networks is often referred to as \textbf{Network Science}. This field was born from what is known as the "Small World Problem,"
discussed in such landmark papers as Stanley Milgram's 1967 thought experiment of the same name. This problem is best represented by the following question:
"Given any two people in the world, person X and person Z, how many intermediate acquaintance links are required before X and Z are connected?" (Milgram, 1967). This experiment has been
pondered extensively, and even carried out by multiple universities, as can be seen in Columbia University's "An Experimental Study of Search in Global Social Networks" (Ribiero 2020). In this experiment,
searchers tasked around 60,000 individuals with contacting one of 18 individuals from 13 different countries by forwarding a single email to acquaintances. Analyzing the path these emails took,
these Columbia researchers were able to map out an extensive network of interconnected individuals. This is Network Science\textemdash the science of identifying the structure of networks and extrapolating information
from them.

\subsection{What is a Network?}
A \textbf{Network} is any set of objects or concepts connected by some relationship or interaction. 
This profoundly vague definition is best expounded on through examples. In the social sciences, a network might be a group of individuals that are related by family, friendship, work, location, academic co-authorship or any other social bond. 
In biology, the interconnected parts of the brain or even the set of proteins responsible for metabolism are networks. 
The hierarchy of objects and classes in an object-oriented programming project is a network. 
Even words can be conceived of as a network when connected by meaning, part of speech, or pronunciation. As can be seen, a network is an abstract tool to envision 
interconnected groups of objects, individuals, and even ideas.   

\subsection{What Can Networks Teach Us?}
The analysis of networks can yield valuable for a multitude of fields. Studying networks can help identify the organizational principles behind them\textemdash patterns, properties, etc.. These organizational principles can form the basis for statistical models
that allow for a better understanding of the behaviors and relationships present in the network. These models essentially map out expected relationships and behaviors for a given group, enabling easy identification of outlier occurences. Finally, most pertinent to this paper,
networks can be represented with mathematical \textbf{Graph Theory} and analyzed with a suite of algorithms utilized for \textbf{Community Detection}.

\section{Graph Theory}

\subsection{What is Graph Theory?}
Graph theory is the study of \textbf{Graphs}, which are the mathematical representations of the real systems referred to as "networks." A more formal definition is given below:

\bigskip

\noindent \textbf{Definition 1:} A \textbf{graph} $G(V, E)$ has two parameters, $V$ and $E$. 

\indent \begin{description}
  \item[\textbf{$V$}] - The set of all vertices, also known as \textbf{nodes}. Nodes represent individual members of a network\textemdash people, roads, products, websites, objects in a piece of software\textemdash which will also be referred to in this paper as agents of a network.
  \item[\textbf{$E$}] - The set of all \textbf{edges}, which connect vertices in the graph. 
\end{description}

\bigskip

\noindent Nodes and Edges can also have any combination of various attributes (Ribeiro 2020). Some examples for edges are:

\bigskip
\indent \begin{description}
  \item[\textbf{Weight}] - Duration of a phone call, distance of a road\dots
  \item[\textbf{Ranking}] - Best Friend, Second Best Friend\dots
  \item[\textbf{Type}] - Friend, Relative, Co-worker (Often represented with different edge colors)  
\end{description}

\bigskip

\noindent Nodes can also have attributes such as:
\bigskip
\begin{description}
  \item[\textbf{Type}] - Nationality, Age, Sex (Often represented with different node colors)
\end{description}
\bigskip

Additionally, there is a set of terms used to describe the number of edges originating from a given node.

\bigskip

\textbf{Definition 2:} The \textbf{degree} of a give node is the total number of outgoing or incoming edges incident on it. 
The degree can be further separated into \textbf{outdegree}, the number of edges originating from the node, and \textbf{indegree},
the number of edges that terminate at the node.

\subsection{Adjacency Matrices}
We will define adjacency matrices, which are common algebraic representations of graphs in Graph Theory (Biggs 7).
Recall from earlier that a graph $G$ has a set of vertices or nodes $V$, and a set of edges $E$ to connect them.
These matrices show which nodes $v \in V$ are adjacent, or connected, to each other.

\bigskip

\textbf{Definition 2:} The \textbf{adjacency matrix} of a network is the $n \times n$ matrix $A$ whose entries $a_{i,j}$ are given by
\[   
a_{i,j} = 
     \begin{cases}
       1 & \text{if node } v_i \text{ and node } v_j \text{ are adjacent, or connected by an edge} \\
       0 & \text{otherwise} \\
     \end{cases}
\]

\bigskip

An example graph, along with its adjacency matrix, is shown here:

\bigskip

\noindent\begin{minipage}{.5\textwidth}
\centering
\includegraphics[scale=0.4]{"s3graph.pdf"}
\captionof{figure}{Graph $G_1$ with three nodes}
\label{fig:fig1}            
\end{minipage}%
\begin{minipage}{.5\textwidth}
\centering
\vspace{1.2cm}
$\begin{bmatrix}
0 & 1 & 1\\
1 & 0 & 1\\
1 & 1 & 0
\end{bmatrix}$
\vspace{1.1cm}

\captionof{figure}{Adjacency matrix $A_1$}
\label{fig:fig2}            
\end{minipage}

\bigskip

The graph $G_1$ has three nodes, with $V = \{ v_1, v_2, v_3\}$.
Each column and row in the adjacency matrix corresponds to a node, so column $1$ and row $1$ correspond to $v_1$, and so on.
As $v_1$ is adjacent (connected) to $v_2$ and $v_3$, entries $a_{1,2}$ and $a_{1,3}$ in adjacency matrix $A_1$ have entry $1$. 
A similar process applies for nodes $v_2$ and $v_3$. 
Notice how all diagonal entries are $0$ as each node is not connected with itself.

\bigskip 
\noindent A more complex example is shown below:

\bigskip

\noindent\begin{minipage}{.5\textwidth}
\centering
\includegraphics[scale=0.4]{s4graph.drawio.pdf}
\captionof{figure}{Graph $G_2$ with four nodes}
\label{fig:fig3}            
\end{minipage}%
\begin{minipage}{.5\textwidth}
\centering
\vspace{1cm}
 $\begin{bmatrix}
0 & 1 & 1 & 0\\
1 & 0 & 0 & 1\\
1 & 0 & 0 & 1\\
0 & 1 & 1 & 0
\end{bmatrix}$
\vspace{0.88cm}

\captionof{figure}{Adjacency matrix $A_2$}
\label{fig:fig4}            
\end{minipage}

\bigskip

\noindent Notice in this example that there is no connection between nodes $v_1, v_4$, and $v_2, v_3$ and thus the associated entries in the adjacency matrix are $0$.

\bigskip 
A few generalizations can be made. The first is that by construction of the adjacency matrix, they will always be square matrices.
Furthermore, the dimensions of the $n \times n$ matrix are that $|V| = n$, as there is a column and row for each node $v \in V$.
In the context of community network graphs, which we are representing in this paper, connections between an node and itself are often omitted for sake of clarity and/or insignificance to the study.
Thus, the diagonal of our adjacency matrices will always be zero. Additionally, these matrices will always be symmetric, as connections (relationships) between nodes are symmetric.

\bigskip 

The complexity of adjacency matrices increases with the presence of weighted graphs and directed graphs which change the aforementioned properties and definitions. 
Although these modifications are significant to real-world studies, they are beyond the scope of this project, which will focus on unweighted, undirected graphs.


\bigskip 


\subsection{Degree Matrices}

\bigskip

While it is important to consider which nodes in a graph are connected through adjacency matrices, another important metric, especially for community detection, is how many connections a particular node has.
This is where degree matrices come in, which represent how many edges are connected to a particular node $v_i \in V$, or essentially how many adjacent nodes $v_i$ has (Chung, Lu, Vu 100).

\bigskip 

\textbf{Definition 4:} The degree matrix of a network is the $n \times n$ matrix $D$ whose entries $d_{i,j}$ are given by
\[   
d_{i,j} = 
     \begin{cases}
       \text{deg} $(v_i)$ & \text{if } i = j \\
       0 & \text{otherwise} \\
     \end{cases}
\]
In this case, deg$(v_i)$ is the total number of edges that are connected to the node $v_i$, per Definition (2) from earlier.

\bigskip 

An example graph, along with its degree matrix, is shown here:

\bigskip

\noindent\begin{minipage}{.5\textwidth}
\centering
\includegraphics[scale=0.4]{s4graph.drawio.pdf}
\captionof{figure}{Graph $G_2$ with four nodes}
\label{fig:fig7}            
\end{minipage}%
\begin{minipage}{.5\textwidth}
\centering
\vspace{1cm}
 $\begin{bmatrix}
2 & 0 & 0 & 0\\
0 & 2 & 0 & 0\\
0 & 0 & 2 & 0\\
0 & 0 & 0 & 2
\end{bmatrix}$
\vspace{0.88cm}

\captionof{figure}{Degree matrix $D_2$}
\label{fig:fig8}            
\end{minipage}

\bigskip

Notice how in the graph, each node $v\in V$ has exactly two connections and thus every entry $d_{i, j}$ along the diagonal has entry $2$.


\bigskip

\noindent A more complex example is shown below.

\bigskip

\noindent\begin{minipage}{.5\textwidth}
\centering
\includegraphics[scale=0.4]{s6graph.drawio.pdf}
\captionof{figure}{Graph $G_3$ with six nodes}
\label{fig:fig5}            
\end{minipage}%
\begin{minipage}{.5\textwidth}
\centering
\vspace{0.5cm}
$\begin{bmatrix}
2 & 0 & 0 & 0 & 0 & 0\\
0 & 3 & 0 & 0 & 0 & 0\\
0 & 0 & 2 & 0 & 0 & 0\\
0 & 0 & 0 & 3 & 0 & 0\\
0 & 0 & 0 & 0 & 3 & 0\\
0 & 0 & 0 & 0 & 0 & 1
\end{bmatrix}$
\vspace{0.5cm}

\captionof{figure}{Degree matrix $D_3$}
\label{fig:fig6}            
\end{minipage}

\bigskip



\bigskip 

\section{Community Detection}

\subsection{What is Community Detection?}

Community Detection is the process by which researchers can identify groups within a network in which individuals' membership in particular cliques are yet to be ascertained. 
Community Detection is performed by researchers to map out the relationships between isolated nodes, members, or objects within the community. 
By gathering and ordering network members by the strength of their relationships to each other one can discover communities\textemdash otherwise known as groups, clusters, cohesive subgroups, or modules. 
This information can yield valuable insights. For example, community detection can reveal the key figures and leaders within a group, around which other members gather.
The results of community detection are often represented through graphs of nodes and edges, and are generated through the application of algorithms to networks as represented by adjacency matrices.

\subsection{CD Algorithm}
There are many different algorithms used for community detection on graphs, each with their own approaches and required data structures. 
For the purposes of this project, we will focus on an algorithm that utilizes the eigenvectors of matrices. 
Specifically, this algorithm tries to find densely connected subgraphs in a graph by calculating the leading non-negative eigenvector of the modularity matrix of the graph (Chandal, Kabre, Aggarwal 66). 

\bigskip 

\noindent This algorithm focuses on modularity, which is a measure the strength of division of a network into modules (also called groups, clusters or communities). The modularity matrix of the graph is an important aspect of this algorithm to detect communities. 


\section{Sample Community Network}
In this simple example, we have interviewed five people about who they eat with most frequently. 
They will rank their peers from 1-4, with 1 representing the person they would be least likely to eat with and 4 representing the person they would most be likely to eat with. 
On a five by five matrix, with zeroes or ones on the diagonal depending on if a person eats with themselves or not, we will convert these values to an adjacency matrix.
This matrix will display the strength of connection according to how often they eat with someone.
We will proceed to generate a directed graph with the use of this data and the CD Algorithm.

\pagebreak
\section*{References}

\bigskip

\noindent “Communities in Networks” by Mason A. Porter, Jukka-Pekka Onnela, and Peter J. Mucha

\bigskip 

\noindent "An Introduction to Community Detection" by Lei Tang and Huan Liu

\bigskip 

\noindent "Community Detection Algorithms" by Thamindu Dilshan Jayawickrama

\bigskip 

\noindent "Algebraic Graph Theory" by Norman Biggs

\bigskip 

\noindent "Lists, Decisions and Graphs. With an Introduction to Probability" by Edward A. Bender and S. Gill Williamson

\bigskip 

\noindent "Community Structure Detection" by Amar Chandole, Ameya Kabre, and \linebreak Atishay Aggarwal

\bigskip

\noindent "An Introduction to Network Science" by Pedro Ribeiro

\bigskip

\noindent Milgram, S. "The small world problem." Psychol. Today 2, 60–67 (1967)

\bigskip

\noindent "Two Decades of Network Science: as seen through the co-authorship network of network scientists" by Roland Molontay and Marcell Nagy

\bigskip 

\noindent "Spectra of random graphs with given expected degrees" by Fan Chung, Linyuan Lu, and Van Vu



%http://web.cs.ucla.edu/~yzsun/classes/2017Winter_CS249/Slides/Clustering1.pdf

%http://ai-nlp.info.uniroma2.it/basili/didattica/BigData/005_Intro_to_CommunityDetection.pdf
%finding community detection https://www.sciencedirect.com/science/article/pii/S1568494616300242
%https://www.nature.com/articles/s41598-018-23932-z
%https://towardsdatascience.com/community-detection-algorithms-9bd8951e7dae
%https://superoles.files.wordpress.com/2015/09/n-biggs-algebraic-graph-theory-1993.pdf
%https://stackoverflow.com/questions/9471906/what-are-the-differences-between-community-detection-algorithms-in-igraph
%https://www.dcc.fc.up.pt/~pribeiro/aulas/ns2021/1_introduction.pdf
%http://files.diario-de-bordo-redes-conecti.webnode.com/200000013-211982212c/AN%20EXPERIMENTAL%20STUDY%20by%20Travers%20and%20Milgram.pdf
%https://arxiv.org/pdf/1908.08478.pdf

\end{document}